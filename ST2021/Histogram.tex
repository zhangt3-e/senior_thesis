%\documentclass[12pt]{article}
%\usepackage[margin=1in]{geometry} 
%\usepackage{amsmath,amsthm,amssymb,amsfonts}
%\usepackage{graphicx}
%\usepackage{float} 
%\newcommand{\N}{\mathbb{N}}
%\newcommand{\Z}{\mathbb{Z}}
%\newenvironment{problem}[2][Problem]{\begin{trivlist}
%		\item[\hskip \labelsep {\bfseries #1}\hskip \labelsep {\bfseries #2.}]}{\end{trivlist}}
%\begin{document}

\begin{figure}[h]
\centering
\includegraphics[0.75\textwidth]{../Figures/intensity_hist_5min.png}
\caption{\label{abc}Distribution of intensity of precipitation events in 2019,
defined as the total precipitation divided by the duration. This
distribution was derived from the distribution of duration of
precipitation with a minimum duration of 5 minutes. The distribution
decreases logarithmically from 0.01 mm/minute to 0.5 mm/minute.} 
\end{figure}
\vfill
\begin{figure}[h]
\centering
\includegraphics[0.75\textwidth]{../Figures/intensity_hist_1min.png}
\caption{\label{abcd}Distribution of intensity of precipitation events in 2019,
defined as the total precipitation divided by the duration. This
distribution was derived from the distribution of duration of
precipitation with a minimum duration of 1 minute. The distribution
decreases logarithmically from 0.01 mm/minute to 0.5 mm/minute.} 
\end{figure}
\vfill
\begin{figure}[h]
\centering\includegraphics[0.75\textwidth]{../Figures/precip_hist_5min.png} 
\caption{\label{abce}Distribution
of duration of precipitation events in 2019. 5 minutes was the minimum
duration needed to define a precipitation event. The distribution is
decreasing logarithmically from the highest values in the 5 minute
precipitation events and the lowest values approaching 100 minutes.}
\end{figure}
\begin{figure}[h]
\centering
\includegraphics[0.75\textwidth]{../Figures/precip_hist_1min.png}
\caption{\label{abcf}A histogram that shows the duration of precipitation
event. Note that in this histogram that the 1 minute was the minimum
duration needed to define a precipitation event. As expected, the
distribution is that we have most precipitation events be close to the
minimum duration and that less precipitation events are particularly
long. } 
\end{figure}
\begin{figure}[h]
\centering \includegraphics[0.75\textwidth]{../Figures/nonprecip_hist_5min.png} 
\caption{\label{abcg}This is a histogram for the duration of a
non-precipitation event, which is to say the gap between two
precipitation events. It also follows the pattern of having lots of
the non-precipitation events be close to the minimum non-precipitation
event of 5 minutes. It does look like that there are more
non-precipitation events that lasts longer than say 40 minutes
compared to the precipitation events. }
\end{figure}
\begin{figure}[h]
\centering
\includegraphics[0.75\textwidth]{../Figures/nonprecip_hist_1min.png}
\caption{\label{abch}This is a histogram for the duration of a non-precipitation
event, which is to say the gap between two precipitation events. Most
events do seem to lie close to the minimum duration of 1 minute.} 
\end{figure}
\begin{figure}[h]
\centering 
\includegraphics[0.75\textwidth]{../Figures/nonprecip1mm_season_19.png} 
\caption{\label{abci}Distribution of the duration of non-precipitation
events separated by seasons. The distribution within each season does
indeed decrease exponentially as we go from 1 minute duration to about
40 minute duration, with the extreme 98th to 100th percentile
excluded.}
\end{figure}
\begin{figure}[h]
\centering
\includegraphics[0.75\textwidth]{../Figures/precip1mm_season_19.png}
\caption{\label{abcj}Distribution of the duration of precipitation events
separated by seasons. The distribution for each season does decrease
exponentially from 1 minute to 40 minute durations. It does seem like
the more precipitation events are closer to the minimum precipitation
duration compared to the non-precipitation events.} 
\end{figure}
\begin{figure}[h]
\centering \includegraphics[0.75\textwidth]{../Figures/inten1mm_season_19.png} 
\caption{\label{abck}Distribution of intensity of precipitation events
separated by seasons. The distribution decrease for each season from
0.01 mm/minute to 0.27 mm/day.}
\end{figure}
\begin{figure}[h]
\centering
\includegraphics[0.75\textwidth]{../Figures/inten1mm_season_19_log.png}
\caption{\label{abcl}Shows the previous figure in terms of log scale for both x
and y axis. It shows that winter does not have very intense
precipitation events and that despite Summer and Winter having similar
amounts of precipitation, (232 mm for Summer to 240 mm for Winter),
summer seems to have more intense precipitation events.}
\end{figure}
%\end{document}
