\documentclass[11pt]{report}
\clubpenalty=10000
\widowpenalty=10000

% It is handy to define new commands for text that occurs frequently (see Discussion)
\newcommand{\MT}{^{\mathrm{MT}}}
\newcommand{\ga}{\gtrsim}
\newcommand{\Lpot}{(L+1)^2}
\newcommand{\WS}{^{\mathrm{WS}}}

%--Format the section headers

\usepackage{amsmath}
\usepackage{amsfonts}
\usepackage{amssymb}
\usepackage{wasysym}
\usepackage{graphicx}
\usepackage{pslatex}
\usepackage{lscape}
\usepackage[T1]{fontenc}
\usepackage[latin1]{inputenc}
\usepackage{longtable}
 \setlength{\LTcapwidth}{5.5 in}
\usepackage{chapterbib}
\usepackage{fancyhdr} % for better header layout
\usepackage{eucal}
\usepackage[english]{babel}
\usepackage[usenames, dvipsnames]{color}
\usepackage[perpage]{footmisc}
\usepackage[round, sort, numbers, authoryear]{natbib}
%\usepackage{multicol} % for pages with multiple text columns, e.g. References
\setlength{\columnsep}{20pt} % space between columns; default 10pt quite narrow
\usepackage[nottoc]{tocbibind} % correct page numbers for bib in TOC, nottoc suppresses an entry for TOC itself
\usepackage{geometry}
\usepackage{setspace}
\usepackage{url}
\usepackage{lastpage}
% FJS Changed this... I didn't like the numbering or the
% indentation... so I introduced a fake chapter Main Text. 
\setcounter{secnumdepth}{0}
\setcounter{tocdepth}{5}

%--set the page formatting--
\geometry{hmargin={1.6in,1.1in},vmargin={1.5in,1.2in}}
\doublespacing

\begin{document}
%--front matter needs roman pagination--
\pagenumbering{roman}

%--Title Page--
\thispagestyle{empty}
  \begin{center}
    \textsc{\LARGE Evaluating Forecasting Methods for\\ Precipitation from Weather Data on top of Guyot Hall } %Fill in your information
  \end{center}
  \vspace{.6in}
  \begin{center}
      Tyrone Zhang
  \end{center}
  \vspace{.6in}
  \begin{center}
    \textsc{A Senior Thesis \\ %Fill in your information
    Presented to the Faculty \\
    of Princeton University \\
    in Candidacy for the Degree \\
    of Bachelor of Arts}
  \end{center}
  \vspace{.3in}
  \begin{center}
    \textsc{Recommended for Acceptance \\
    by the \\Department of  Geosciences \\}
    Adviser: Frederik J.~Simons
  \end{center}
  \vspace{.3in}
  \begin{center}
  \today
  \end{center}
  
  \clearpage


%--Copyright Page--
\thispagestyle{empty}
\vspace*{3in}
\begin{center}
\emph{This paper represents my own work in accordance with University regulations,} \\
Your Signature %%Sign here
\end{center}
\clearpage

%--Abstract--  
\addcontentsline{toc}{chapter}{Abstract}
\begin{center}
\Large \textbf{Abstract}
\end{center}
 
% Senior thesis or Junior Project Abstract -----------------------------------------------------

Princeton's climate has four seasons, with strong temperature
variations, and precipitation occurring throughout the
year. Statistically, precipitation event sequences can be
characterized as drawn from exponential distributions in the three
variables the precipitation event \textit{duration}, \textit{intensity} 
(the total precipitation divided by the duration), and the non-precipitation 
\textit{duration}. The shortest and least intense precipitation events 
are the most frequent. Analyzing the precipitation measured from 2017 
to the present day by a Vaisala WXT530 weather station located on the 
roof of Guyot Hall, I first summarize the data in terms
of exponential distributions and their parameters, by season and by
year. Subsequently, I evaluate the skill in predicting the arrival,
duration and intensity of precipitation events solely based on this
local ``climatology'', before including other variables logged by the
weather station. Predicting precipitation events using this climatology
yielded 2-3 \% precipitation accuracy. Thus, we proceeded to use linear
regression and decision trees regression to improve the precipitation
accuracy. Linear regression yielded 4.7 \%, while decision tree yielded 
11.9\%. Then neural networks were used in the form of LSTM, where we had 
hourly and minute inputs. The hourly input
resulted in 9.9\%, while the minute inputs resulted in 16.8\%. With each new model, 
we are able to see improvements in the accuracy of
predicting precipitation. However, there are further improvements that can be 
made with many pathways forward to improve the precipitation accuracy.  

 \clearpage

%--Acknowledgements--  
\addcontentsline{toc}{chapter}{Acknowledgements}
\begin{center}
\Large \textbf{Acknowledgements}
\end{center}

% Senior thesis or Junior Project Acknowledgements  -----------------------------------------------------

%Delete the text below and write your acknowledgements
I would like to acknowledge my senior thesis advisor Frederik J. Simons for giving me constant feedback on my work as well as providing me with the data that he is collecting on top of Guyot Hall. 
\clearpage

%--Table of Contents--  
\thispagestyle{empty}
\tableofcontents
\clearpage

\listoffigures 
\listoftables
\clearpage

%--Set up fancy header-- 
\fancyhead{}
\fancyfoot{}
\pagestyle{fancyplain}
\rhead{\fancyplain{\thepage}{\noindent \textsc{\rightmark} \hfill \thepage~of~\pageref{LastPage}}}
\rfoot{\hrule \today \hfill Your Name}
\pagenumbering{arabic}

%--Reset the page numbers and set them to arabic-- 
{\newpage\renewcommand{\thepage}{\arabic{page}}\setcounter{page}{1}}

%--Have sections but use chapter counters
\addcontentsline{toc}{chapter}{Main Text}

\section{Introduction \label{sec:introduction}}

% \documentclass[12pt]{article}
\usepackage[margin=1in]{geometry} 
\usepackage{amsmath,amsthm,amssymb,amsfonts}
\usepackage{graphicx}
\usepackage{float}
\newcommand{\N}{\mathbb{N}}
\newcommand{\Z}{\mathbb{Z}}
\newenvironment{problem}[2][Problem]{\begin{trivlist}
		\item[\hskip \labelsep {\bfseries #1}\hskip \labelsep {\bfseries #2.}]}{\end{trivlist}}
\begin{document}
	\begin{figure}[h]
		\centering
		\includegraphics[width=150mm]{intensity_hist_5min.png}
		\caption{This is a histogram of intensity of precipitation events, in which the intensity is the total precipitation of a precipitation event divided by the duration of the precipitation event. We can see the distribution decrease logarithmically as we got from 0.01 mm/minute to 0.5 mm/minute in intensity.}
	\end{figure}
	\begin{figure}[h]
	\centering
	\includegraphics[width=150mm]{intensity_hist_1min.png}
	\caption{This is a histogram of intensity of precipitation events in which the minimum precipitation event duration is set to 1 minute, in which the intensity is the total precipitation of a precipitation event divided by the duration of the precipitation event. It is clear to see lots of intensity of precipitation events are near 0.01 mm/minute.}
\end{figure}
\begin{figure}[h]
	\centering
	\includegraphics[width=150mm]{precip_hist_5min.png}
	\caption{A histogram that shows the duration of precipitation event. Note that in this histogram that the 5 minutes was the minimum duration needed to define a precipitation event. As expected, the distribution is that we have most precipitation events be close to the minimum duration and that less precipitation events are particularly long. }
\end{figure}
\begin{figure}[h]
	\centering
	\includegraphics[width=150mm]{precip_hist_1min.png}
	\caption{A histogram that shows the duration of precipitation event. Note that in this histogram that the 1 minute was the minimum duration needed to define a precipitation event. As expected, the distribution is that we have most precipitation events be close to the minimum duration and that less precipitation events are particularly long. }
\end{figure}
\begin{figure}[h]
	\centering
	\includegraphics[width=150mm]{nonprecip_hist_5min.png}
	\caption{This is a histogram for the duration of a non-precipitation event, which is to say the gap between two precipitation events. It also follows the pattern of having lots of the non-precipitation events be close to the minimum non-precipitation event of 5 minutes. It does look like that there are more non-precipitation events that lasts longer than say 40 minutes compared to the precipitation events. }
\end{figure}
\begin{figure}[h]
	\centering
	\includegraphics[width=150mm]{nonprecip_hist_1min.png}
	\caption{This is a histogram for the duration of a non-precipitation event, which is to say the gap between two precipitation events. Most events do seem to lie close to the minimum duration of 1 minute. }
\end{figure}
\end{document}

\section{Methods \label{sec:methods}}
I shall define the following terms, $E_j^\tau $ is the term to define a precipitation event, which consists of precipitation that has fallen over a time period. The precipitation event has $e_i$, a continuous set of non-zeros precipitation, flanked left and right by 0s. The minimum duration in order to define a precipitation event is $\tau$, which is the minimum duration in minutes. Furthermore we can define a non-precipitation event $N_k^{\tau_1}$, which has a continuous set of zeros, $z_i$, flanked left and right by non-zero terms. A minimum duration for a non-precipitation event is $\tau_1$, which is the minimum duration in minutes. One more term to define is intensity, which for a precipitation event is the total amount of precipitation divided by the duration of the event, which will be denoted as $I_j = \frac{A_j}{d_j}$, in which $A_j = \int P(t) dt$ gives us the total precipitation of a precipitation event. 
\\ 
For further analysis, breaking the year down into seasons since different seasons will have different characteristics with regards to precipitation. I will define the seasons as follows: Winter will be December, January, and February. Spring will be March, April, and May. Summer is June, July, and August. Fall is September, October, November. To make sure that we can maintain temporal continuity, December will be pulled from the previous year, so for 2019 Winter, we take December from 2018. 
\begin{figure}[h]
	\centering
	\includegraphics[width=0.8\textwidth]{Figures/More_detail_precip_2019.png}
	\caption{\label{Precipitation histogram for 2019 seasons} Distribution of Precipitation duration in the 2019 year with a break down by seasons. The maximum precipitation represented here is the 98th percentile precipitation duration of the year.  The minimum acceptable duration, $E_j$ is $E_1$, which is one minute.  }
\end{figure}

\section{Results \label{sec:results}}


\section{Discussion \label{sec:discussion}}


\section{Conclusions \label{sec:conclusions}}


%--References
\small
\renewcommand{\bibsep}{0em}

\renewcommand{\bibname}{References}
\bibliographystyle{Latex/gji}
\bibliography{refs}

\end{document}
