\documentclass[11pt]{report}
\clubpenalty=10000
\widowpenalty=10000

% It is handy to define new commands for text that occurs frequently (see Discussion)
\newcommand{\MT}{^{\mathrm{MT}}}
\newcommand{\ga}{\gtrsim}
\newcommand{\Lpot}{(L+1)^2}
\newcommand{\WS}{^{\mathrm{WS}}}
\newcommand{\fracd}[2]{\frac{\displaystyle{#1}}{\displaystyle{#2}}} 

%--Format the section headers

\usepackage{amsmath}
\usepackage{amsfonts}
\usepackage{amssymb}
\usepackage{wasysym}
\usepackage{graphicx}
\usepackage{pslatex}
\usepackage{lscape}
\usepackage[T1]{fontenc}
\usepackage[latin1]{inputenc}
\usepackage{longtable}
 \setlength{\LTcapwidth}{5.5 in}
\usepackage{chapterbib}
\usepackage{fancyhdr} % for better header layout
\usepackage{eucal}
\usepackage[english]{babel}
\usepackage[usenames, dvipsnames]{color}
\usepackage[perpage]{footmisc}
\usepackage[round, sort, numbers, authoryear]{natbib}
%\usepackage{multicol} % for pages with multiple text columns, e.g. References
\setlength{\columnsep}{20pt} % space between columns; default 10pt quite narrow
\usepackage[nottoc]{tocbibind} % correct page numbers for bib in TOC, nottoc suppresses an entry for TOC itself
\usepackage{geometry}
\usepackage{setspace}
\usepackage{url}
\usepackage{lastpage}
% FJS Changed this... I didn't like the numbering or the
% indentation... so I introduced a fake chapter Main Text. 
\setcounter{secnumdepth}{0}
\setcounter{tocdepth}{5}

%--set the page formatting--
\geometry{hmargin={1.6in,1.1in},vmargin={1.5in,1.2in}}
\doublespacing

\begin{document}
%--front matter needs roman pagination--
\pagenumbering{roman}

%--Title Page--
\thispagestyle{empty}
  \begin{center}
    \textsc{\LARGE Evaluating Forecasting Methods for\\ Precipitation from Weather Data on top of Guyot Hall } %Fill in your information
  \end{center}
  \vspace{.6in}
  \begin{center}
      Tyrone Zhang
  \end{center}
  \vspace{.6in}
  \begin{center}
    \textsc{A Senior Thesis \\ %Fill in your information
    Presented to the Faculty \\
    of Princeton University \\
    in Candidacy for the Degree \\
    of Bachelor of Arts}
  \end{center}
  \vspace{.3in}
  \begin{center}
    \textsc{Recommended for Acceptance \\
    by the \\Department of  Geosciences \\}
    Adviser: Frederik J.~Simons
  \end{center}
  \vspace{.3in}
  \begin{center}
  \today
  \end{center}
  
  \clearpage


%--Copyright Page--
\thispagestyle{empty}
\vspace*{3in}
\begin{center}
\emph{This paper represents my own work in accordance with University regulations,} \\
Tyrone Zhang %%Sign here
\end{center}
\clearpage

%--Abstract--  
\addcontentsline{toc}{chapter}{Abstract}
\begin{center}
\Large \textbf{Abstract}
\end{center}
 
% Senior thesis or Junior Project Abstract -----------------------------------------------------

Princeton's climate has four seasons, with strong temperature
variations, and precipitation occurring throughout the
year. Statistically, precipitation event sequences can be
characterized as drawn from exponential distributions in the three
variables the precipitation event \textit{duration}, \textit{intensity} 
(the total precipitation divided by the duration), and the non-precipitation 
\textit{duration}. The shortest and least intense precipitation events 
are the most frequent. Analyzing the precipitation measured from 2017 
to the present day by a Vaisala WXT530 weather station located on the 
roof of Guyot Hall, I first summarize the data in terms
of exponential distributions and their parameters, by season and by
year. Subsequently, I evaluate the skill in predicting the arrival,
duration and intensity of precipitation events solely based on this
local ``climatology'', before including other variables logged by the
weather station. Predicting precipitation events using this climatology
yielded 2-3 \% precipitation accuracy. Thus, we proceeded to use linear
regression and decision trees regression to improve the precipitation
accuracy. Linear regression yielded 4.7 \%, while decision tree yielded 
11.9\%. Then neural networks were used in the form of LSTM, where we had 
hourly and minute inputs. The hourly input
resulted in 9.9\%, while the minute inputs resulted in 16.8\%. With each new model, 
we are able to see improvements in the accuracy of
predicting precipitation. However, there are further improvements that can be 
made with many pathways forward to improve the precipitation accuracy.  

 \clearpage

%--Acknowledgements--  
\addcontentsline{toc}{chapter}{Acknowledgements}
\begin{center}
\Large \textbf{Acknowledgements}
\end{center}

% Senior thesis or Junior Project Acknowledgements  -----------------------------------------------------

%Delete the text below and write your acknowledgements
I would like to acknowledge my senior thesis advisor Frederik J. Simons for giving me constant feedback on my work as well as providing me with the data that he is collecting on top of Guyot Hall. 
\clearpage

%--Table of Contents--  
\thispagestyle{empty}
\tableofcontents
\clearpage

\listoffigures 
\listoftables
\clearpage

%--Set up fancy header-- 
\fancyhead{}
\fancyfoot{}
\pagestyle{fancyplain}
\rhead{\fancyplain{\thepage}{\noindent \textsc{\rightmark} \hfill \thepage~of~\pageref{LastPage}}}
\rfoot{\hrule \today \hfill Tyrone Zhang}
\pagenumbering{arabic}

%--Reset the page numbers and set them to arabic-- 
{\newpage\renewcommand{\thepage}{\arabic{page}}\setcounter{page}{1}}

%--Have sections but use chapter counters
\addcontentsline{toc}{chapter}{Main Text}

\section{Introduction \label{sec:introduction}}

The climatology of Princeton is one that belongs to the mid-latitudes,
which is characterized by having four seasons that results in a large
variation in temperature throughout a year. In terms of the average
calculated between 1981 and 2010, Princeton gets an average of 1227~mm
of precipitation annually, and the precipitation distribution
throughout the year is fairly even, with less precipitation in the
winter~\cite[]{PRISM}.  According to the Koppen-Geiger Climate
Classification, Princeton, NJ lies in the classification Cfa, which
denotes a temperate climate, with no dry season, and hot summers
defined as reaching~22$^\circ$C or higher \cite[]{Peel2008}. Princeton
having no dry season means that precipitation is well spread out
throughout the year.

% \documentclass[12pt]{article}
\usepackage[margin=1in]{geometry} 
\usepackage{amsmath,amsthm,amssymb,amsfonts}
\usepackage{graphicx}
\usepackage{float}
\newcommand{\N}{\mathbb{N}}
\newcommand{\Z}{\mathbb{Z}}
\newenvironment{problem}[2][Problem]{\begin{trivlist}
		\item[\hskip \labelsep {\bfseries #1}\hskip \labelsep {\bfseries #2.}]}{\end{trivlist}}
\begin{document}
	\begin{figure}[h]
		\centering
		\includegraphics[width=150mm]{intensity_hist_5min.png}
		\caption{This is a histogram of intensity of precipitation events, in which the intensity is the total precipitation of a precipitation event divided by the duration of the precipitation event. We can see the distribution decrease logarithmically as we got from 0.01 mm/minute to 0.5 mm/minute in intensity.}
	\end{figure}
	\begin{figure}[h]
	\centering
	\includegraphics[width=150mm]{intensity_hist_1min.png}
	\caption{This is a histogram of intensity of precipitation events in which the minimum precipitation event duration is set to 1 minute, in which the intensity is the total precipitation of a precipitation event divided by the duration of the precipitation event. It is clear to see lots of intensity of precipitation events are near 0.01 mm/minute.}
\end{figure}
\begin{figure}[h]
	\centering
	\includegraphics[width=150mm]{precip_hist_5min.png}
	\caption{A histogram that shows the duration of precipitation event. Note that in this histogram that the 5 minutes was the minimum duration needed to define a precipitation event. As expected, the distribution is that we have most precipitation events be close to the minimum duration and that less precipitation events are particularly long. }
\end{figure}
\begin{figure}[h]
	\centering
	\includegraphics[width=150mm]{precip_hist_1min.png}
	\caption{A histogram that shows the duration of precipitation event. Note that in this histogram that the 1 minute was the minimum duration needed to define a precipitation event. As expected, the distribution is that we have most precipitation events be close to the minimum duration and that less precipitation events are particularly long. }
\end{figure}
\begin{figure}[h]
	\centering
	\includegraphics[width=150mm]{nonprecip_hist_5min.png}
	\caption{This is a histogram for the duration of a non-precipitation event, which is to say the gap between two precipitation events. It also follows the pattern of having lots of the non-precipitation events be close to the minimum non-precipitation event of 5 minutes. It does look like that there are more non-precipitation events that lasts longer than say 40 minutes compared to the precipitation events. }
\end{figure}
\begin{figure}[h]
	\centering
	\includegraphics[width=150mm]{nonprecip_hist_1min.png}
	\caption{This is a histogram for the duration of a non-precipitation event, which is to say the gap between two precipitation events. Most events do seem to lie close to the minimum duration of 1 minute. }
\end{figure}
\end{document}

Our weather station on the top of Guyot Hall is Vaisala weather
transmitter WXT530 series. It measures six weather parameters of air
pressure, temperature, humidity, rainfall, wind speed, and wind
direction. The rainfall is measured using an acoustic Vaisala RAINCAP
Sensor, which helps avoid the complications of flooding, wetting, and
evaporation losses \cite[]{Vaisala}. By analyzing the precipitation
bwteen 2017 and today, I first summarize the data using this
climatology, in terms of exponential distributions and their
parameters, by season and by year, before attepting to predict
precipitation events based on other variables that are observed by the weather station.

These following histograms show the distribution of precipitation event duration in terms of minutes and shows the different duration distributions per season. 

\clearpage
\begin{figure}[t]
	\centering
	\includegraphics[width=0.675\textwidth]{Figures/precip_2017.png}
	\caption[Precipitation histogram for 2017 broken down by
	season]{\label{p2017} Distribution of precipitation duration in
		2017, with a breakdown by seasons. Note that data collection began on 16 July
		2017. Which means we only have part of the summer and all of the fall data. }
\end{figure}
\begin{figure}[b]
	\centering
	\includegraphics[width=0.675\textwidth]{Figures/precip_2018.png}
	\caption[Precipitation histogram for 2018 broken down by season]{\label{p2018}
		Distribution of precipitation duration in 2018, with a breakdown
		by seasons. 2018 is the first full year of data collections.}
\end{figure}

\clearpage
\begin{figure}[t]
	\centering
	\includegraphics[width=0.675\textwidth]{Figures/precip_2019.png}
	\caption[Precipitation histogram for 2019 broken down by
	season]{\label{p2019} Distribution of precipitation duration in
		2019, with a breakdown by seasons. %The minimum duration,
		%$\tau=1$~min, and the maximum duration over the data set is 368	min, but the horizontal axis was limited to the 98th percentile of
		%the durations, 42~min, for clarity. Superposed is a least-squares
		%fit of a line in log-log space of duration versus frequency for the entire year, excluding the 1--2~min bin, quoted as the
		%equivalent exponential in this space.
	}
\end{figure}

\begin{figure}[b]
	\centering
	\includegraphics[width=0.675\textwidth]{Figures/precip_2020.png}
	\caption[Precipitation histogram for 2020 broken down by
	season]{\label{p2020} Distribution of precipitation duration
		in 2020, with a breakdown by seasons.  %The layout is as in
		%Figure~\ref{p2019}. 
	}
\end{figure}

\clearpage

\begin{figure}[t]
	\centering
	\includegraphics[width=0.675\textwidth]{Figures/nonprecip_2017.png}
	\caption[Histogram of non-precipitation events for 2017 broken down by
	season]{\label{np2017} Distribution of non-precipitation event duration in	2017, with a breakdown by seasons. The minimum duration, $\tau = 1$~min. Note that data collection began on 16 July 2017, which means that Summer is only partially complete.}
\end{figure}
\begin{figure}[b]
	\centering
	\includegraphics[width=0.675\textwidth]{Figures/nonprecip_2018.png}
	\caption[Histogram of non-precipitation events for 2018 broken down by season]{\label{np2018}
		Distribution of precipitation duration in 2018, with a breakdown
		by seasons. The minimum duration, $\tau=1$~min. }
\end{figure}

\clearpage
\begin{figure}[t]
	\centering
	\includegraphics[width=0.675\textwidth]{Figures/nonprecip_2019.png}
	\caption[Histogram of non-precipitation events for 2019 broken
	down by season]{\label{np2019} Distribution of
		non-precipitation event duration in 2019, with a breakdown
		by seasons. The minimum duration, $\tau=1$~min %, and the
		%maximum duration over the data set is 2613~min, but the
		%horizontal axis was limited to the 98th percentile of
		%2350~min, for clarity. Superposed is a least-squares fit of
		%a line in log-log space of duration versus frequency for the
		%entire year, excluding the 1--2~min bin, quoted as the
		%equivalent exponential in this space.
	}
\end{figure}

\begin{figure}[b]
	\centering
	\includegraphics[width=0.675\textwidth]{Figures/nonprecip_2020.png}
	\caption[Histogram of non-precipitation events for 2020 broken down by
	season]{\label{np2020} Distribution of non precipitation event duration
		in 2020, with a breakdown by seasons. }
\end{figure}


\clearpage
\begin{figure}[t]
	\centering
	\includegraphics[width=0.675\textwidth]{Figures/inten2017.png}
	\caption[Intensity histogram for 2017 broken down by season]
	{\label{i2017}Distribution of precipitation intensity in 2017,
		with a breakdown by seasons. Note that data collection began on 16
		July 2017.}
\end{figure}
\begin{figure}[b]
	\centering
	\includegraphics[width=0.675\textwidth]{Figures/inten2018.png}
	\caption[Intensity histogram for 2018 broken down by season]
	{\label{i2018}Distribution of precipitation intensity in 2018,
		with a breakdown by seasons.}
\end{figure}
\clearpage
\begin{figure}[t]
	\centering
	\includegraphics[width=0.675\textwidth]{Figures/inten2019.png}
	\caption[Intensity histogram for 2019 broken down by season]
	{\label{i2019}Distribution of precipitation intensity in
		2019, with a breakdown by seasons. The minimum intensity,
		$I=0.01$~mm/min. 
	}
\end{figure}
\begin{figure}[b]
	\centering
	\includegraphics[width=0.675\textwidth]{Figures/inten2020.png}
	\caption[Intensity histogram for 2020 broken down by season]
	{\label{i2020} Distribution of precipitation intensity in
		2020, with a breakdown by seasons.}
\end{figure}
\clearpage


\section{Methods \label{sec:methods}}

I shall define the following terms. The time series of
\textbf{precipitation} as recorded by the instrument is denoted $e_i$,
where $i$ indexes the measurement intervals, each 60~s long. I define
a precipitation \textbf{event} $E_j^\tau $ as a sequence of
\textbf{duration} $d_j\ge \tau$ containing contiguous nonzero
precipitation measurements $e_i>0$, flanked left and right by zeros,
$e_i=0$, and where $\tau$ is in minutes.

Furthermore, I define a precipitation \textbf{non-event} $N_j^\tau$,
  as having a contiguous set of zeros, $e_i=0$, whose combined duration
  exceeds $\tau$, flanked left and right by non-zero values, $e_i=0$.

One more term to define is \textbf{precipitation intensity}, which for
a precipitation event $E_j^\tau$ is the total amount of precipitation
divided by its duration, i.e., 
\begin{equation}
I_j^\tau = \fracd{\sum_i e_i }{d_j} ,
\quad
\mbox{for}\,\,\,\, i\,\,\,\, \mbox{belonging to the event}\,\,\,\, E_j^\tau
.
\end{equation}



For further analysis, I breaking down the year into seasons, as
different seasons may have different characteristics with regards to
precipitation. I will define the seasons as follows: Winter will be
December, January, and February. ``Winter'' of a certain year contains
December of the previous year.  Spring will be March, April, and
May. Summer is June, July, and August. Fall is September, October,
November.

Once we have characterized the climatology of precipitation, we will start moving towards buidling predictive models for precipitation. In order to see how accurate such models are in comparison to observed precipitation. Such accuracy we shall use is to see whether the model and the observed data match in terms of whether there exists precipitation at a given minute. We ignore the non-precipitation when looking at accuracy because by matching non-precipitation minutes in both the model and the observed data, we end up getting an accuracy of over 90\%, which not useful. So, if the model and the observed data do not match in terms whether there exists precipitation, this contributes to the model being deemed less accurate. We can have a stricter definition of accuracy, in which we set up the precipitation condition, in addition to saying that the intensity must match too, otherwise we can not say the model and observed data match. This stricter definition of model accuracy might be used when thinking about 


Figure~\ref{p2019} shows the distribution of durations of 3198
precipitation events $E_j^1$, i.e. $E_j^\tau$ where $\tau=1$~min for
the year 2019, broken down by season. In order to make bins that contain non-zero values, I created bins using duration percentiles. I used unique values obtained from using a range of percentiles from 0 to 100, in 2 and 5 percent intervals. For one analysis, I stopped at 98 \% believing this would be the best approach in terms of fitting. At the same time, I also made sure to analyze the data including the 100 percentile, to see how it differs when including the extremes. Such purpose is also to realize that excluding such extremes produced modelled precipitation that did not last long as well as well as the gaps between precipitation events being too small.  

I used an exponential fit to the
frequency-duration histograms for all 1253 events $E_j^2$,
i.e. $E_j^\tau$ where $\tau=2$~min. For all the other years, as shown
in Figures~\ref{p2020}, \ref{p2018} and~\ref{p2017}, I used a similar
procedure.

Excluding the first interval shown, focusing on events of duration
greater than or equal to 2~min, we propose an exponential model for
the histogram, with the following equation:
\begin{equation}\label{expod}
  F = \beta \,e^{\alpha d},
\end{equation}
where $F$ is the frequency and $d$ the duration, and with $\beta$ the
unitless frequency coefficient and $\alpha$ is the exponential
coefficient (in units of min$^{-1}$). Table~\ref{firsttable} shows the
coefficients~$\beta$ and the exponential coefficients~$\alpha$ from
looking at the yearly frequency of precipitation duration.

We shall also propose the following equations which will also describe an exponential model for the histogram regarding non-precipitation events, which is described by the following equation: 
\begin{equation}\label{expod_np}
	F_{np} = \gamma \,e^{\delta D},
\end{equation}
where $F_{np}$ is the frequency of non-precipitation events, $D$ being duration, and $\gamma $ being the unitless frequency coefficient and $\delta $ is the exponential coefficient. Table~\ref{thirdtable_98} shows the coefficients~$\gamma$ and exponential coefficients~$\delta$ from looking at yearly frequency of non-precipitation event durations. 

Another similar equation for precipitation intensity can be described by the following equation: 
\begin{equation}\label{expod_inten}
	F_{inten} = \epsilon \,e^{\zeta I},
\end{equation}
where $F_{inten}$ is the frequency of intensity of precipitation events, $I$ is the intensity of the precipitation events, $\epsilon$ is the unitless frequency coefficient, and $\zeta$ is the exponential coefficient. Table~\ref{fourthtable} shows the coefficients~$\epsilon$ and the exponential coefficients~$\zeta$ from looking at yearly frequency of intensity of precipitation events.  

\clearpage
\begin{figure}[t]
	\centering
	\includegraphics[width=0.625\textwidth]{Figures/precip17_new.png}
	\caption[2017 Exponentials with contrasting curve fitting]
	{\label{precip17_redone}
		Shows curve fitting of the histogram excluding 1 minute duration events for the incomplete 2017 data. The red curve denotes the curve that fits to the 100th percentile, while the black curve fits the data to the 98th percentile. Has similar trends to the other years, though 2017 has incomplete data.   }
\end{figure}
\begin{figure}[b]
	\centering
	\includegraphics[width=0.625\textwidth]{Figures/precip18_new.png}
	\caption[2018 Exponentials with contrasting curve fitting]
	{\label{precip18_redone}
		Shows curve fitting of the histogram excluding 1 minute duration events for the entire 2018 data. The red curve denotes the curve that fits to the 100th percentile, while the black curve fits the data to the 98th percentile. The 98 percentile curve fitting the smaller durations better, while the 100 percentile curve fitting the larger durations better.  Perhaps the only real difference between 2017 and 2018 is the frequency, with 2018 having complete data.  }
\end{figure}

\clearpage

\begin{figure}[t]
	\centering
	\includegraphics[width=0.675\textwidth]{Figures/precip19_new.png}
	\caption[2019 Exponentials with contrasting curve fitting]
	{\label{precip19_redone}
		Shows curve fitting of the histogram excluding 1 minute duration events for the entire 2019 data. The layout is the same as seen in \ref{precip18_redone}. 2018 and 2019 have similar trends and similar curves for both best fit curves.  }
\end{figure}
\begin{figure}[b]
	\centering
	\includegraphics[width=0.675\textwidth]{Figures/precip20_new.png}
	\caption[2020 Exponentials with contrasting curve fitting]
	{\label{precip20_redone}
		Shows curve fitting of the histogram excluding 1 minute duration events for the entire 2020 data. The layout is the same as seen in \ref{precip18_redone}. 2018, 2019, and 2020 all look similar looking at the histogram and the best fit curves.   }
\end{figure}
\clearpage
%Is the exponential different for W/S/S/F? Can you tell the difference?
%Is the exponential different for different years? Can you tell the
%difference?  And what summarizes it all. 

\begin{table}[htb]
  \begin{center}
    \begin{tabular}{|l|*{11}{r|}r|}
      \hline
      Season    &       \multicolumn{2}{|c|}{Annual}          & \multicolumn{2}{|c|}{Winter}& \multicolumn{2}{|c|}{Spring}  & \multicolumn{2}{|c|}{Summer} &\multicolumn{2}{|c|}{Fall}  \\
      \hline
      Year      & $\beta $ & $\alpha$  & $\beta $ & $\alpha$ & $\beta $ & $\alpha$ & $\beta $ & $\alpha$ & $\beta $ & $\alpha$\\
      \hline
      2017      & \textit{169}  & \textit{-0.61}  & NaN & NaN & NaN & NaN & \textit{57}  & \textit{-0.70}  & 108  & -0.57  \\
      2018      & 392           & -0.59  & 148 & -0.74 & 125 & -0.76 & 65  & -0.52  & 91 & -0.50  \\
      2019      & 308           & -0.54  & 118  & -0.62 & 107 & -0.58 & 32 & -0.31  & 57 &  -0.60 \\
      2020      & 408           & -0.65   & 201  & -0.80 & 112  & -0.66 & 48  & -0.42 & \textit{64} & \textit{-0.66}\\
      \hline
    \end{tabular}
  \end{center}
\caption[Year comparison of
  coefficients of precipitation duration using 0 to 98th percentile]{\label{firsttable}Coefficients found from the yearly
  distribution of precipitation duration (as in equation~\ref{expod})
  as well as the seasonal distribution of precipitation
  duration. Italics refer to values obtained using incomplete
  information. NaN means there was no information. These coefficients were computed using the 0 to 98th percentile of precipitation duration.}
\end{table}


\begin{table}[bh]
  \begin{center}
    \begin{tabular}{|l|*{11}{r|}r|}
      \hline
      Season    &       \multicolumn{2}{|c|}{Annual}          & \multicolumn{2}{|c|}{Winter}& \multicolumn{2}{|c|}{Spring}  & \multicolumn{2}{|c|}{Summer} &\multicolumn{2}{|c|}{Fall}  \\
      \hline
      Year      & $\beta $ & $\alpha$  & $\beta $ & $\alpha$ & $\beta $ & $\alpha$ & $\beta $ & $\alpha$ & $\beta $ & $\alpha$\\
      \hline
      2017      & \textit{94}  & \textit{-0.32}  & NaN & NaN & NaN & NaN & \textit{32}  & \textit{-0.14}  & 68  & -0.09  \\
      2018      & 214           & -0.28  & 73 & -0.37 & 46 & -0.27 & 38  & -0.24  & 55 & -0.23  \\
      2019      & 224          & -0.28  & 68  & -0.34 & 60 & -0.28 & 25 & -0.19  & 33 &  -0.32 \\
      2020      & 234           & -0.35   & 119  & -0.51 & 60  & -0.32 & 36  & -0.27 & \textit{29} & \textit{-0.23}\\
      \hline
    \end{tabular}
  \end{center}
\caption[Year comparison of coefficients of precipitation duration up 
  to its 100th percentile]{\label{firsttable_100} Coefficients found from the
  yearly distribution of precipitation duration (as in equation~\ref{expod})
  as well as the seasonal distribution of precipitation duration. Italics
  refer to values obtained using incomplete information. NaN means there was
  no information. These coefficients were computed using the all the data,
  from 0 to 100th percentile.}
\end{table}

Based on Table~\ref{firsttable}, the $\beta$ values are similar to
each other with the exception of 2017, which only had partial data
starting in the Summer. Even, with the partial data we have from 2017
and 2020, it is clear that the $\alpha$ values are similar to each
other. Focusing on the years with complete data, 2018 and 2019, it is
clear that yearly variations exists between them. All $\alpha$ values
are negative.

In the seasonal variations, we see that Summer has $\alpha$ values
that are less negative compared to the other seasons as well as having
a lower $\beta$ compared to the other seasons. However looking at
Table~\ref{secondtable}, we see that the summers also have a lot of
precipitation. The fewer amounts of precipitation events in the
Summer, but with a lot of precipitation gives a higher intensity for
the Summer. For the Winter, the $\alpha$ are the furthest from 0 and
the $\beta$ are large. However, Winter tends to have the lowest values
for total precipitation and the combination of lots of preciptation
events and low precipitation totals results in the lowest intensities
among the four seasons.

The Spring appears to mimic the Winter in that there are a lot of
preciptiation events, but also shares the quality of Summer in having
fairly high precipitation totals. The Fall shares the Summer quality
of having relatively few preciptation events, but tends to have
smaller precipitation totals, so its intensity is less than the
Summer, but greater than the winter precipitation intensity.
\begin{table}[t]
  \begin{center}
    \begin{tabular}{|l|*{11}{c|}r|}
      \hline
      Season    &       \multicolumn{2}{|c|}{Annual}          & \multicolumn{2}{|c|}{Winter}& \multicolumn{2}{|c|}{Spring}  & \multicolumn{2}{|c|}{Summer} &\multicolumn{2}{|c|}{Fall}  \\
      \hline
      Year      & T & I  & T & I  & T & I  & T & I  & T & I \\
      \hline
      2017      & \textit{520}  & \textit{0.07}  & NaN & NaN & NaN & NaN & \textit{131}  & \textit{0.07}  & 388  & 0.07  \\
      2018      & 1104           & 0.06  & 147 & 0.03 & 301 & 0.07 & 268  & 0.08  & 388 & 0.07  \\
      2019      & 1016           & 0.06  & 218  & 0.04 & 318 & 0.06 & 352 & 0.11  & 127 &  0.05 \\
      2020      & \textit{858}           & \textit{0.06}   & 151  & 0.03 & 191  & 0.04 & 322  & 0.12 & \textit{194} & \textit{0.06}\\
      \hline
    \end{tabular}
  \end{center}
  \caption[Summary of total precipitation and
    intensity]{\label{secondtable}Total precipitation (in mm) and average
    intensity (in mm/min) of precipitation for each year and season. Italics
    refer to values obtained using incomplete information. NaN means there
    was no information.}
\end{table}
 

Looking at trying to extend the exponential fit towards non-precipitation event durations, the exponentials for such duration, $\delta$ is less negative overall compared to the exponentials from the precipitation event duration, $\alpha$. Some of this is explained from the fact that the duration of non-precipitation events range from 1 minute to over 2000 minutes, whereas the duration of precipitation events range from 1 minute to just over 300 minutes. At the same time how we bin the durations for both precipitation and non-precipitation events will affect how we get the fits. 






\begin{table}[htb]
  \begin{center}
    \begin{tabular}{|l|*{11}{c|}r|}
      \hline
      Season    &       \multicolumn{2}{|c|}{Annual}          & \multicolumn{2}{|c|}{Winter}& \multicolumn{2}{|c|}{Spring}  & \multicolumn{2}{|c|}{Summer} &\multicolumn{2}{|c|}{Fall}  \\
      \hline
      Year      & $\gamma $ & $\delta$  & $\gamma $ & $\delta$ & $\gamma $ & $\delta$ & $\gamma $ & $\delta$ & $\gamma $ & $\delta$\\
      \hline
      2017      & \textit{33}  & \textit{-0.13}  & NaN & NaN & NaN & NaN &   &   &   &   \\
      2018      & 189           & -0.20  &  &  &  &  &   &   &  &   \\
      2019      & 136           & -0.16  &  &  &  &  &  &  &  &   \\
      2020      & 140           & -0.17  &  &  &  &  &   &  &  & \\
      \hline
    \end{tabular}
  \end{center}
  \caption[Year comparison of coefficients for non-precipitation
    events] {\label{thirdtable}Coefficients found from the yearly
    distribution of non-precipitation event duration (as in
    equation~\ref{}) as well as the seasonal distribution of
    non-precipitation event duration. Italics refer to values obtained
    using incomplete information. NaN means there was no
    information. }
\end{table}

\begin{table}[htb]
  \begin{center}
    \begin{tabular}{|l|*{11}{c|}r|}
      \hline
      Season    &       \multicolumn{2}{|c|}{Annual}          & \multicolumn{2}{|c|}{Winter}& \multicolumn{2}{|c|}{Spring}  & \multicolumn{2}{|c|}{Summer} &\multicolumn{2}{|c|}{Fall}  \\
      \hline
      Year      & $\gamma $ & $\delta$  & $\gamma $ & $\delta$ & $\gamma $ & $\delta$ & $\gamma $ & $\delta$ & $\gamma $ & $\delta$\\
      \hline
      2017      & \textit{30}  & \textit{-0.10}  & NaN & NaN & NaN & NaN & \textit{11}  & \textit{-0.10}  & 18  & -0.09  \\
      2018      & 160           & -0.14  & 47 & -0.19 & 43 & -0.14 & 23  & -0.05  & 46 & -0.15  \\
      2019      & 120           & -0.11  & 49 & -0.21 & 44 & -0.15 & 11 & 0.04 & 19 & -0.08   \\
      2020      & 121           & -0.11  & 57 & -0.19 & 41 & -0.12 & 12  & 0.01  & 15 & -0.11 \\
      \hline
    \end{tabular}
  \end{center}
  \caption[Year comparison of coefficients for non-precipitation
    events for 0 to 100th percentile] {\label{thirdtable_100}Coefficients found from the yearly
    distribution of non-precipitation event duration (as in
    equation~\ref{expod_np}) as well as the seasonal distribution of
    non-precipitation event duration. Italics refer to values obtained
    using incomplete information. NaN means there was no
    information. Using non-precipitation event durations from the 0 to 100th percentile.}
\end{table}

\begin{table}[htb]
  \begin{center}
    \begin{tabular}{|l|*{11}{c|}r|}
      \hline
      Season    &       \multicolumn{2}{|c|}{Annual}          & \multicolumn{2}{|c|}{Winter}& \multicolumn{2}{|c|}{Spring}  & \multicolumn{2}{|c|}{Summer} &\multicolumn{2}{|c|}{Fall}  \\
      \hline
      Year      & $\epsilon $ & $\zeta$  &  $\epsilon $ & $\zeta$  &  $\epsilon $ & $\zeta$  &  $\epsilon $ & $\zeta$  & $\epsilon $ & $\zeta$ \\
      \hline
      2017      & \textit{14}  & \textit{-0.07}  & NaN & NaN & NaN & NaN &   &   &   &   \\
      2018      & 20           & -0.47  &  &  &  &  &   &   &  &   \\
      2019      & 24           & -0.34  &  &  &  &  &  &  &  &   \\
      2020      & 37          & -0.18  &  &  &  &  &   &  &  & \\
      \hline
    \end{tabular}
  \end{center}
  \caption[Year comparison of coefficients for precipitation
    intensity] {\label{fourthtable}Coefficients found from the yearly
    distribution of precipitation intensity (as in equation~\ref{}) as
    well as the seasonal distribution of precipitation
    intensity. Italics refer to values obtained using incomplete
    information. NaN means there was no information. }
\end{table}

\begin{table}[htb]
  \begin{center}
    \begin{tabular}{|l|*{11}{c|}r|}
      \hline
      Season    &       \multicolumn{2}{|c|}{Annual}          & \multicolumn{2}{|c|}{Winter}& \multicolumn{2}{|c|}{Spring}  & \multicolumn{2}{|c|}{Summer} &\multicolumn{2}{|c|}{Fall}  \\
      \hline
      Year      & $\epsilon $ & $\zeta$  &  $\epsilon $ & $\zeta$  &  $\epsilon $ & $\zeta$  &  $\epsilon $ & $\zeta$  & $\epsilon $ & $\zeta$ \\
      \hline
      2017      & \textit{1.6}  & \textit{-1.05}  & NaN & NaN & NaN & NaN & \textit{0.8}  & \textit{-0.94}  & 0.8  & -1.12 \\
      2018      & 12           & -0.919  & 0.7 & -1.36 & 3 & -0.91  & 6  & -0.67  & 4 & -0.91  \\
      2019      & 9           & -0.94  & 0.10 & -1.9 & 3 & -0.94 & 7 & -0.48 & 1 & -1.0  \\
      2020      & 16          & -0.76  & 0.5 & -1.4 & 3 & -0.86 & 13  & -0.32 & 3 & -0.69\\
      \hline
    \end{tabular}
  \end{center}
  \caption[Year comparison of coefficients for precipitation
    intensity using 0 to 100 percentile] {\label{fourthtable_100}Coefficients found from the yearly
    distribution of precipitation intensity (as in equation~\ref{expod_inten}) as
    well as the seasonal distribution of precipitation
    intensity. Italics refer to values obtained using incomplete
    information. NaN means there was no information. Using intensity from 0 to 100th percentile. }
\end{table}


In contrast, Table~\ref{fourthtable} for intensity of precipitation events show a more negative exponential compared to either precipitation or non-precipitation event durations. This is partially due to the fact that there are less bins in the intensity of preciptiation events as seen that the vast majority of precipitation events are not particularly intense on average. 



\clearpage
\section{Results\label{sec:results}}
\begin{figure}[t]
	\centering
	\includegraphics[width=0.75\textwidth]{Figures/better_one_run.png}
	\caption[One run using Summer 2018 climatology]
	{\label{crudemodel}One model run using the exponentials calculated from Summer 2018.  }
\end{figure}
Running the model one hundred times let us look at what the average precipitation total the model yields, it yields 195 mm compared to the Summer 2018 total of 268 mm that we used. Furthermore, the accuracy of model precipitation matching model precipitation is only 6$\%$. Looking at \ref{crudemodel}, we see that the model precipitation total is lower than the actual precipitation observed in the Summer of 2018. Furthermore, the precipitation seem to be more frequent compared to the observed 2018 Summer data. 

\clearpage
\begin{figure}[t]
	\centering
	\includegraphics[width=0.75\textwidth]{Figures/best_one_run.png}
	\caption[Modified run using Summer 2018 climatology]
	{\label{crudermodel} One model run using the exponentials calculated from Summer 2018. Made some adjustments to get a better match for the actual summer 2018 run.}
\end{figure}

\clearpage
\begin{figure}[t]
	\centering
	\includegraphics[width=0.75\textwidth]{Figures/run_with_more_info.png}
	\caption[More  run using Summer 2018 climatology]
	{\label{crudesmodel} One model run using the exponentials calculated from Summer 2018. Put up some more information such as how much of the time was non-dominated by no precipitation. }
\end{figure}


\section{Discussion\label{sec:discussion}}
This section is blank for now. 

\section{Conclusions\label{sec:conclusions}}
This section is blank for now. 

%--References
\small
\renewcommand{\bibsep}{0em}

\renewcommand{\bibname}{References}
\bibliographystyle{Latex/gji}
\bibliography{refs}

\end{document}
