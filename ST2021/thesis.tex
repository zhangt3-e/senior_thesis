\documentclass[11pt]{report}
\clubpenalty=10000
\widowpenalty=10000

% It is handy to define new commands for text that occurs frequently (see Discussion)
\newcommand{\MT}{^{\mathrm{MT}}}
\newcommand{\ga}{\gtrsim}
\newcommand{\Lpot}{(L+1)^2}
\newcommand{\WS}{^{\mathrm{WS}}}
\newcommand{\fracd}[2]{\frac{\displaystyle{#1}}{\displaystyle{#2}}} 

%--Format the section headers

\usepackage{amsmath}
\usepackage{amsfonts}
\usepackage{amssymb}
\usepackage{wasysym}
\usepackage{graphicx}
\usepackage{pslatex}
\usepackage{lscape}
\usepackage[T1]{fontenc}
\usepackage[latin1]{inputenc}
\usepackage{longtable}
 \setlength{\LTcapwidth}{5.5 in}
\usepackage{chapterbib}
\usepackage{fancyhdr} % for better header layout
\usepackage{eucal}
\usepackage[english]{babel}
\usepackage[usenames, dvipsnames]{color}
\usepackage[perpage]{footmisc}
\usepackage[round, sort, numbers, authoryear]{natbib}
%\usepackage{multicol} % for pages with multiple text columns, e.g. References
\setlength{\columnsep}{20pt} % space between columns; default 10pt quite narrow
\usepackage[nottoc]{tocbibind} % correct page numbers for bib in TOC, nottoc suppresses an entry for TOC itself
\usepackage{geometry}
\usepackage{setspace}
\usepackage{url}
\usepackage{lastpage}
% FJS Changed this... I didn't like the numbering or the
% indentation... so I introduced a fake chapter Main Text. 
\setcounter{secnumdepth}{0}
\setcounter{tocdepth}{5}

%--set the page formatting--
\geometry{hmargin={1.6in,1.1in},vmargin={1.5in,1.2in}}
\doublespacing

\begin{document}
%--front matter needs roman pagination--
\pagenumbering{roman}

%--Title Page--
\thispagestyle{empty}
  \begin{center}
    \textsc{\LARGE Evaluating Forecasting Methods for\\ Precipitation from Weather Data on top of Guyot Hall } %Fill in your information
  \end{center}
  \vspace{.6in}
  \begin{center}
      Tyrone Zhang
  \end{center}
  \vspace{.6in}
  \begin{center}
    \textsc{A Senior Thesis \\ %Fill in your information
    Presented to the Faculty \\
    of Princeton University \\
    in Candidacy for the Degree \\
    of Bachelor of Arts}
  \end{center}
  \vspace{.3in}
  \begin{center}
    \textsc{Recommended for Acceptance \\
    by the \\Department of  Geosciences \\}
    Adviser: Frederik J.~Simons
  \end{center}
  \vspace{.3in}
  \begin{center}
  \today
  \end{center}
  
  \clearpage


%--Copyright Page--
\thispagestyle{empty}
\vspace*{3in}
\begin{center}
\emph{This paper represents my own work in accordance with University regulations,} \\
Tyrone Zhang %%Sign here
\end{center}
\clearpage

%--Abstract--  
\addcontentsline{toc}{chapter}{Abstract}
\begin{center}
\Large \textbf{Abstract}
\end{center}
 
% Senior thesis or Junior Project Abstract -----------------------------------------------------

Princeton's climate has four seasons, with strong temperature
variations, and precipitation occurring throughout the
year. Statistically, precipitation event sequences can be
characterized as drawn from exponential distributions in the three
variables the precipitation event \textit{duration}, \textit{intensity} 
(the total precipitation divided by the duration), and the non-precipitation 
\textit{duration}. The shortest and least intense precipitation events 
are the most frequent. Analyzing the precipitation measured from 2017 
to the present day by a Vaisala WXT530 weather station located on the 
roof of Guyot Hall, I first summarize the data in terms
of exponential distributions and their parameters, by season and by
year. Subsequently, I evaluate the skill in predicting the arrival,
duration and intensity of precipitation events solely based on this
local ``climatology'', before including other variables logged by the
weather station. Predicting precipitation events using this climatology
yielded 2-3 \% precipitation accuracy. Thus, we proceeded to use linear
regression and decision trees regression to improve the precipitation
accuracy. Linear regression yielded 4.7 \%, while decision tree yielded 
11.9\%. Then neural networks were used in the form of LSTM, where we had 
hourly and minute inputs. The hourly input
resulted in 9.9\%, while the minute inputs resulted in 16.8\%. With each new model, 
we are able to see improvements in the accuracy of
predicting precipitation. However, there are further improvements that can be 
made with many pathways forward to improve the precipitation accuracy.  

 \clearpage

%--Acknowledgements--  
\addcontentsline{toc}{chapter}{Acknowledgements}
\begin{center}
\Large \textbf{Acknowledgements}
\end{center}

% Senior thesis or Junior Project Acknowledgements  -----------------------------------------------------

%Delete the text below and write your acknowledgements
I would like to acknowledge my senior thesis advisor Frederik J. Simons for giving me constant feedback on my work as well as providing me with the data that he is collecting on top of Guyot Hall. 
\clearpage

%--Table of Contents--  
\thispagestyle{empty}
\tableofcontents
\clearpage

\listoffigures 
\listoftables
\clearpage

%--Set up fancy header-- 
\fancyhead{}
\fancyfoot{}
\pagestyle{fancyplain}
\rhead{\fancyplain{\thepage}{\noindent \textsc{\rightmark} \hfill \thepage~of~\pageref{LastPage}}}
\rfoot{\hrule \today \hfill Tyrone Zhang}
\pagenumbering{arabic}

%--Reset the page numbers and set them to arabic-- 
{\newpage\renewcommand{\thepage}{\arabic{page}}\setcounter{page}{1}}

%--Have sections but use chapter counters
\addcontentsline{toc}{chapter}{Main Text}

\section{Introduction \label{sec:introduction}}
The Climatology of Princeton is clearly one that belongs to the mid-latitudes, which is characterized by having four seasons that results in having a large variation in temperature throughout a year. In terms of the average calculated between 1981 and 2010, Princeton gets an average of 48.27 inches (1227 mm) of precipitation annually, and the precipitation distribution throughout the year is fairly even, with less precipitation in the winter (PRISM Project).  According to the Koppen-Geiger Climate Classification, Princeton, NJ lies in the classification Cfa, which denotes a temperate climate, with no dry season, and hot summers defined as reaching 22 $^\circ C $ or higher \cite{Peel2008}. Princeton having no dry season means that precipitation should be well spread throughout the year. 

% \documentclass[12pt]{article}
\usepackage[margin=1in]{geometry} 
\usepackage{amsmath,amsthm,amssymb,amsfonts}
\usepackage{graphicx}
\usepackage{float}
\newcommand{\N}{\mathbb{N}}
\newcommand{\Z}{\mathbb{Z}}
\newenvironment{problem}[2][Problem]{\begin{trivlist}
		\item[\hskip \labelsep {\bfseries #1}\hskip \labelsep {\bfseries #2.}]}{\end{trivlist}}
\begin{document}
	\begin{figure}[h]
		\centering
		\includegraphics[width=150mm]{intensity_hist_5min.png}
		\caption{This is a histogram of intensity of precipitation events, in which the intensity is the total precipitation of a precipitation event divided by the duration of the precipitation event. We can see the distribution decrease logarithmically as we got from 0.01 mm/minute to 0.5 mm/minute in intensity.}
	\end{figure}
	\begin{figure}[h]
	\centering
	\includegraphics[width=150mm]{intensity_hist_1min.png}
	\caption{This is a histogram of intensity of precipitation events in which the minimum precipitation event duration is set to 1 minute, in which the intensity is the total precipitation of a precipitation event divided by the duration of the precipitation event. It is clear to see lots of intensity of precipitation events are near 0.01 mm/minute.}
\end{figure}
\begin{figure}[h]
	\centering
	\includegraphics[width=150mm]{precip_hist_5min.png}
	\caption{A histogram that shows the duration of precipitation event. Note that in this histogram that the 5 minutes was the minimum duration needed to define a precipitation event. As expected, the distribution is that we have most precipitation events be close to the minimum duration and that less precipitation events are particularly long. }
\end{figure}
\begin{figure}[h]
	\centering
	\includegraphics[width=150mm]{precip_hist_1min.png}
	\caption{A histogram that shows the duration of precipitation event. Note that in this histogram that the 1 minute was the minimum duration needed to define a precipitation event. As expected, the distribution is that we have most precipitation events be close to the minimum duration and that less precipitation events are particularly long. }
\end{figure}
\begin{figure}[h]
	\centering
	\includegraphics[width=150mm]{nonprecip_hist_5min.png}
	\caption{This is a histogram for the duration of a non-precipitation event, which is to say the gap between two precipitation events. It also follows the pattern of having lots of the non-precipitation events be close to the minimum non-precipitation event of 5 minutes. It does look like that there are more non-precipitation events that lasts longer than say 40 minutes compared to the precipitation events. }
\end{figure}
\begin{figure}[h]
	\centering
	\includegraphics[width=150mm]{nonprecip_hist_1min.png}
	\caption{This is a histogram for the duration of a non-precipitation event, which is to say the gap between two precipitation events. Most events do seem to lie close to the minimum duration of 1 minute. }
\end{figure}
\end{document}

 The weather station that I am getting my data from is a Vaisala weather transmitter WXT530 series. It measures the six weather parameters of air pressure, temperature, humidity, rainfall, wind speed ,and wind direction. The rainfall is measured using an acoustic Vaisala RAINCAP Sensor, which helps avoid the complications of flooding, wetting, and evaporation losses \cite{Vaisala}. By analyzing the precipitation that is measured from Professor Simons' Vaisala weather station on the top of Guyot Hall from 2017 to present day, I can first summarize the data that is being characterized, then start using this climatology to start predicting precipitation events based on other variables that are observed in the weather station. 
\section{Methods \label{sec:methods}}

I shall define the following terms. The time series of
\textbf{precipitation} as recorded by the instrument is denoted $e_i$,
where $i$ indexes the measurement intervals, each 60~s long. I define
a precipitation \textbf{event} $E_j^\tau $ as a sequence of
\textbf{duration} $d_j\ge \tau$ containing contiguous nonzero
precipitation measurements $e_i>0$, flanked left and right by zeros,
$e_i=0$, and where $\tau$ is in minutes.

Furthermore, I define a precipitation \textbf{non-event} $N_j^\tau$,
  as having a contiguous set of zeros, $e_i=0$, whose combined duration
  exceeds $\tau$, flanked left and right by non-zero values, $e_i=0$.

One more term to define is \textbf{precipitation intensity}, which for
a precipitation event $E_j^\tau$ is the total amount of precipitation
divided by its duration, i.e., 
\begin{equation}
I_j^\tau = \fracd{\sum_i e_i }{d_j} ,
\quad
\mbox{for}\,\,\,\, i\,\,\,\, \mbox{belonging to the event}\,\,\,\, E_j^\tau
.
\end{equation}


For further analysis, breaking the year down into seasons since
different seasons will have different characteristics with regards to
precipitation. I will define the seasons as follows: Winter will be
December, January, and February. ``Winter'' of a certain year contains
December of the previous year.  Spring will be March, April, and
May. Summer is June, July, and August. Fall is September, October,
November.

Figure~\ref{p2019} shows the distribution of durations of 3198
precipitation events $E_j^1$, i.e. $E_j^\tau$ where $\tau=1$~min for
the year 2019, broken down by season. I used an exponential fit to
the frequency-duration histograms for all 1253 events $E_j^2$,
i.e. $E_j^\tau$ where $\tau=2$~min. For all the other years, I have 
used similar procedure in which the distribution of the duration of 
precipitation events for events $E_j^1$ are graphed and an exponential 
fit is used for events $E_j^2$ and correspondingly graphed. 

\begin{table}[b]
	\begin{center}
		\begin{tabular}{|l|*{11}{c|}r|}
			\hline
			Season    &       \multicolumn{2}{|c|}{Annual}          & \multicolumn{2}{|c|}{Winter}& \multicolumn{2}{|c|}{Spring}  & \multicolumn{2}{|c|}{Summer} &\multicolumn{2}{|c|}{Fall}  \\
			\hline
			Year      & $\beta $ & $\alpha$  & $\beta $ & $\alpha$ & $\beta $ & $\alpha$ & $\beta $ & $\alpha$ & $\beta $ & $\alpha$\\
			\hline
			2017      & \textit{169}  & \textit{-0.61}  & NaN & NaN & NaN & NaN & \textit{57}  & \textit{-0.70}  & 108  & -0.57  \\
			2018      & 392           & -0.59  & 148 & -0.74 & 125 & -0.76 & 65  & -0.52  & 91 & -0.50  \\
			2019      & 308           & -0.54  & 118  & -0.62 & 107 & -0.58 & 32 & -0.31  & 57 &  -0.60 \\
			2020      & 408           & -0.65   & 201  & -0.80 & 112  & -0.66 & 48  & -0.42 & \textit{64} & \textit{-0.66}\\
			\hline
		\end{tabular}
	\end{center}
	\caption[Year comparison of coefficients]{This is the co-efficients found from the yearly distribution of precipitation as well as the seasonl distribution of precipitation. Italics refer to values obtained using incomplete information. NaN means there was no information. }
\end{table}
 
\begin{figure}[t]
	\centering
	\includegraphics[width=0.675\textwidth]{Figures/precip_2020.png}
	\caption[Precipitation histogram for 2020 broken down by season]{\label{p2020}
		Distribution of precipitation duration in 2020, with a breakdown
		by seasons. 2020 is also incomplete as collection still ongoing. The layout is as in Figure~\ref{p2019}.}
\end{figure}
\begin{figure}[b]
  \centering
  \includegraphics[width=0.675\textwidth]{Figures/More_detail_precip_2019.png}
  \caption[Precipitation histogram for 2019 broken down by
    season]{\label{p2019} Distribution of precipitation duration in
    2019, with a breakdown by seasons. The minimum duration,
    $\tau=1$~min, and the maximum duration over the data set is 368
    min, but the horizontal axis was limited to the 98th percentile of
    the durations, 42~min, for clarity. Superposed is a least-squares
    fit of a line in log-log space of duration versus frequency for
    the entire year, excluding the 1--2~min bin, quoted as the
    equivalent exponential in this space.}
\end{figure}
\clearpage
\begin{figure}[t]
  \centering
  \includegraphics[width=0.675\textwidth]{Figures/precip_2018.png}
  \caption[Precipitation histogram for 2018 broken down by season]{\label{p2018}
    Distribution of precipitation duration in 2018, with a breakdown
    by seasons. The layout is as in Figure~\ref{p2019}.}
\end{figure}

\begin{figure}[b]
	\centering
	\includegraphics[width=0.675\textwidth]{Figures/precip_2017.png}
	\caption[Precipitation histogram for 2017 broken down by
          season]{\label{p2017} Distribution of precipitation duration
          in 2017, with a breakdown by seasons. The layout is as in
          Figure~\ref{p2019}. Note that data collection began on July 16th
          of this year.}
\end{figure}


\clearpage

	
%Is the exponential different for W/S/S/F? Can you tell the difference?
%Is the exponential different for different years? Can you tell the
%difference?  And what summarizes it all. 
\begin{table}[t]
	\begin{center}
		\begin{tabular}{|l|*{11}{c|}r|}
			\hline
			Season    &       \multicolumn{2}{|c|}{Annual}          & \multicolumn{2}{|c|}{Winter}& \multicolumn{2}{|c|}{Spring}  & \multicolumn{2}{|c|}{Summer} &\multicolumn{2}{|c|}{Fall}  \\
			\hline
			Year      & T & I  & T & I  & T & I  & T & I  & T & I \\
			\hline
			2017      & \textit{520}  & \textit{0.07}  & NaN & NaN & NaN & NaN & \textit{131}  & \textit{0.07}  & 388  & 0.07  \\
			2018      & 1104           & 0.06  & 147 & 0.03 & 301 & 0.07 & 268  & 0.08  & 388 & 0.07  \\
			2019      & 1016           & 0.06  & 218  & 0.04 & 318 & 0.06 & 352 & 0.11  & 127 &  0.05 \\
			2020      & \textit{858}           & \textit{0.06}   & 151  & 0.03 & 191  & 0.04 & 322  & 0.12 & \textit{194} & \textit{0.06}\\
			\hline
		\end{tabular}
	\end{center}
	\caption[Summary of total precipitation and intensity]{This is the total precipitation and intensity of precipitation for each year and season. Italics refer to values obtained using incomplete information. NaN means there was no information. }
\end{table}
Excluding the first interval shown, focusing on events of duration
greater than or equal to 2 min, we propose an exponential model for
the histogram, with the following equation:
\begin{equation}
	F = \beta e^{\alpha d},
\end{equation}
where $F$ is the frequency and $d$ the duration, and with $\beta$ the
unitless frequency coefficient and $\alpha$ is the exponential
coefficient (in units of 1/min).
Table 1 shows
the coefficients and the exponential coefficients from the looking at
the yearly frequency of precipitation duration. 
\\ Based on this table, the $\beta$ values are similar to each other with
the exception of 2017, which only had partial data starting from the
summer. Even, with the partial data we have from 2017 and 2020, it is
clear that the $\alpha$ values are similar to each other. Focusing on
the years with complete data, 2018 and 2019, it is clear that yearly
variations exists between them. In the seasonal variations, we see that the Summer has $\alpha $ that are closer to 0 compared to the other seasons. 


\section{Results \label{sec:results}}
This section is blank for now. (Quite likely that the stuff in Methods may constitute results)

\section{Discussion \label{sec:discussion}}
This section is blank for now. 

\section{Conclusions \label{sec:conclusions}}
This section is blank for now. 

%--References
\small
\renewcommand{\bibsep}{0em}

\renewcommand{\bibname}{References}
\bibliographystyle{Latex/gji}
\bibliography{refs}

\end{document}
