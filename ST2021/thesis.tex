\documentclass[11pt]{report}
\clubpenalty=10000
\widowpenalty=10000

% It is handy to define new commands for text that occurs frequently (see Discussion)
\newcommand{\MT}{^{\mathrm{MT}}}
\newcommand{\ga}{\gtrsim}
\newcommand{\Lpot}{(L+1)^2}
\newcommand{\WS}{^{\mathrm{WS}}}

%--Format the section headers

%\usepackage{nameref}
\usepackage{amsmath}
\usepackage{amsfonts}
\usepackage{amssymb}
\usepackage{wasysym}
\usepackage{graphicx}
\usepackage{pslatex}
\usepackage{lscape}
\usepackage[T1]{fontenc}
\usepackage[latin1]{inputenc}
\usepackage{longtable}
 \setlength{\LTcapwidth}{5.5 in}
\usepackage{chapterbib}
\usepackage{fancyhdr} % for better header layout
\usepackage{eucal}
\usepackage[english]{babel}
\usepackage[usenames, dvipsnames]{color}
\usepackage[perpage]{footmisc}
\usepackage[round, sort, numbers, authoryear]{natbib}
%\usepackage{multicol} % for pages with multiple text columns, e.g. References
\setlength{\columnsep}{20pt} % space between columns; default 10pt quite narrow
\usepackage[nottoc]{tocbibind} % correct page numbers for bib in TOC, nottoc suppresses an entry for TOC itself
\usepackage{geometry}
\usepackage{setspace}
\usepackage{url}
\usepackage{lastpage}

% FJS Changed this... I didn't like the numbering or the
% indentation... so I introduced a fake chapter Main Text. 
\setcounter{secnumdepth}{0}
\setcounter{tocdepth}{5}

%--set the page formatting--
\geometry{hmargin={1.6in,1.1in},vmargin={1.5in,1.2in}}
\doublespacing

\begin{document}
%--front matter needs roman pagination--
\pagenumbering{roman}

%--Title Page--
\thispagestyle{empty}
  \begin{center}
    \textsc{\LARGE Using Princeton Precipitation climatology to predict future precipitation events} %Fill in your information
  \end{center}
  \vspace{.6in}
  \begin{center}
      Tyrone Zhang
  \end{center}
  \vspace{.6in}
  \begin{center}
    \textsc{A Senior Thesis \\ %Fill in your information
    Presented to the Faculty \\
    of Princeton University \\
    in Candidacy for the Degree \\
    of Bachelor of Arts}
  \end{center}
  \vspace{.3in}
  \begin{center}
    \textsc{Recommended for Acceptance \\
    by the \\Department of  Geosciences \\}
    Adviser: Frederik J.~Simons
  \end{center}
  \vspace{.3in}
  \begin{center}
  \today
  \end{center}
  
  \clearpage


%--Copyright Page--
\thispagestyle{empty}
\vspace*{3in}
\begin{center}
\emph{This paper represents my own work in accordance with University regulations,} \\
Your Signature %%Sign here
\end{center}
\clearpage

%--Abstract--  
\addcontentsline{toc}{chapter}{Abstract}
\begin{center}
\Large \textbf{Abstract}
\end{center}
 
% Senior thesis or Junior Project Abstract -----------------------------------------------------

%Delete the text below and write your abstract
Princeton's climate is one that has four seasons and a high temperature variation through the year. The precipitation in Princeton is spread out throughout the year. Precipitation events are often characterized by an exponential distribution of both the duration and the total precipitation per event. The shortest precipitation events and the smallest precipitation totals are the most frequent, while the longer the precipitation event, the less likely it is to occur at any given point.  By analyzing the precipitation that is measured from Professor Simons' Vaisala weather station on the top of Guyot Hall from 2017 to present day, I can first summarize the data that is being characterized, then start using this climatology to start predicting precipitation events based on other variables that are observed in the weather station. 

 \clearpage

%--Acknowledgements--  
\addcontentsline{toc}{chapter}{Acknowledgements}
\begin{center}
\Large \textbf{Acknowledgements}
\end{center}

% Senior thesis or Junior Project Acknowledgements  -----------------------------------------------------

%Delete the text below and write your acknowledgements
I would like to acknowledge my senior thesis advisor Frederik J. Simons for giving me constant feedback on my work as well as providing me with the data that he is collecting on top of Guyot Hall. His patience and guidance through this tough year was welcomed for sure. I also thank Professor Alan Rubin for being my second reader. 

I also like to acknowledge my family, who has been very supportive and understanding in my time in Princeton, especially during this past year. 

My Princeton friends who were also in the thesis grind who were also struggling over the past year. Our common struggle helped us bond in these rather tough times. 

Finally all those in the Geoscience department, especially my fellow seniors in which we tried to make the best out of a weird situation for our seniors. 
\clearpage

%--Table of Contents--  
\thispagestyle{empty}
\tableofcontents
\clearpage

\listoffigures 
\listoftables
\clearpage

%--Set up fancy header-- 
\fancyhead{}
\fancyfoot{}
\pagestyle{fancyplain}
\rhead{\fancyplain{\thepage}{\noindent \textsc{\rightmark} \hfill \thepage~of~\pageref{LastPage}}}
\rfoot{\hrule \today \hfill Your Name}
\pagenumbering{arabic}

%--Reset the page numbers and set them to arabic-- 
{\newpage\renewcommand{\thepage}{\arabic{page}}\setcounter{page}{1}}

%--Have sections but use chapter counters
\addcontentsline{toc}{chapter}{Main Text}

\section{Introduction \label{sec:introduction}}

% %\documentclass[12pt]{article}
%\usepackage[margin=1in]{geometry} 
%\usepackage{amsmath,amsthm,amssymb,amsfonts}
%\usepackage{graphicx}
%\usepackage{float} 
%\newcommand{\N}{\mathbb{N}}
%\newcommand{\Z}{\mathbb{Z}}
%\newenvironment{problem}[2][Problem]{\begin{trivlist}
%		\item[\hskip \labelsep {\bfseries #1}\hskip \labelsep {\bfseries #2.}]}{\end{trivlist}}
%\begin{document}

\begin{figure}[h]
\centering
\includegraphics0.75\textwidth{../Figures/intensity_hist_5min.png}
\caption{\label{abc}Distribution of intensity of precipitation events in 2019,
defined as the total precipitation divided by the duration. This
distribution was derived from the distribution of duration of
precipitation with a minimum duration of 5 minutes. The distribution
decreases logarithmically from 0.01 mm/minute to 0.5 mm/minute.} 
\end{figure}
\vfill
\begin{figure}[h]
\centering
\includegraphics0.75\textwidth{../Figures/intensity_hist_1min.png}
\caption{\label{abcd}Distribution of intensity of precipitation events in 2019,
defined as the total precipitation divided by the duration. This
distribution was derived from the distribution of duration of
precipitation with a minimum duration of 1 minute. The distribution
decreases logarithmically from 0.01 mm/minute to 0.5 mm/minute.} 
\end{figure}
\vfill
\begin{figure}[h]
\centering\includegraphics0.75\textwidth{../Figures/precip_hist_5min.png} 
\caption{\label{abce}Distribution
of duration of precipitation events in 2019. 5 minutes was the minimum
duration needed to define a precipitation event. The distribution is
decreasing logarithmically from the highest values in the 5 minute
precipitation events and the lowest values approaching 100 minutes.}
\end{figure}
\begin{figure}[h]
\centering
\includegraphics0.75\textwidth{../Figures/precip_hist_1min.png}
\caption{\label{abcf}A histogram that shows the duration of precipitation
event. Note that in this histogram that the 1 minute was the minimum
duration needed to define a precipitation event. As expected, the
distribution is that we have most precipitation events be close to the
minimum duration and that less precipitation events are particularly
long. } 
\end{figure}
\begin{figure}[h]
\centering \includegraphics0.75\textwidth{../Figures/nonprecip_hist_5min.png} 
\caption{\label{abcg}This is a histogram for the duration of a
non-precipitation event, which is to say the gap between two
precipitation events. It also follows the pattern of having lots of
the non-precipitation events be close to the minimum non-precipitation
event of 5 minutes. It does look like that there are more
non-precipitation events that lasts longer than say 40 minutes
compared to the precipitation events. }
\end{figure}
\begin{figure}[h]
\centering
\includegraphics0.75\textwidth{../Figures/nonprecip_hist_1min.png}
\caption{\label{abch}This is a histogram for the duration of a non-precipitation
event, which is to say the gap between two precipitation events. Most
events do seem to lie close to the minimum duration of 1 minute.} 
\end{figure}
\begin{figure}[h]
\centering 
\includegraphics0.75\textwidth{../Figures/nonprecip1mm_season_19.png} 
\caption{\label{abci}Distribution of the duration of non-precipitation
events separated by seasons. The distribution within each season does
indeed decrease exponentially as we go from 1 minute duration to about
40 minute duration, with the extreme 98th to 100th percentile
excluded.}
\end{figure}
\begin{figure}[h]
\centering
\includegraphics0.75\textwidth{../Figures/precip1mm_season_19.png}
\caption{\label{abcj}Distribution of the duration of precipitation events
separated by seasons. The distribution for each season does decrease
exponentially from 1 minute to 40 minute durations. It does seem like
the more precipitation events are closer to the minimum precipitation
duration compared to the non-precipitation events.} 
\end{figure}
\begin{figure}[h]
\centering \includegraphics0.75\textwidth{../Figures/inten1mm_season_19.png} 
\caption{\label{abck}Distribution of intensity of precipitation events
separated by seasons. The distribution decrease for each season from
0.01 mm/minute to 0.27 mm/day.}
\end{figure}
\begin{figure}[h]
\centering
\includegraphics0.75\textwidth{../Figures/inten1mm_season_19_log.png}
\caption{\label{abcl}Shows the previous figure in terms of log scale for both x
and y axis. It shows that winter does not have very intense
precipitation events and that despite Summer and Winter having similar
amounts of precipitation, (232 mm for Summer to 240 mm for Winter),
summer seems to have more intense precipitation events.}
\end{figure}
%\end{document}


\section{Methods \label{sec:methods}}
I shall define the following terms, $P(t)$ is the term to define a precipitation event, which consists of precipitation that has fallen over a time period. The precipitation event has $e_i$, a continuous set of non-zeros, flanked left and right by 0s. The minimum duration in order to define a precipitation event is $E_j$, where $j$ is the minimum duration in minutes. Furthermore we can define a non-precipitation event $N(t)$, which has a continuous set of zeros, $z_i$, flanked left and right by non-zero terms. A minimum duration for a non-precipitation event is $F_k$, where $k$ is the minimum duration in minute. One more term to define is intensity, which for a precipitation event is the total amount of precipitation divided by the duration of the event, which will be denoted as $I$.
\section{Results \label{sec:results}}


\section{Discussion \label{sec:discussion}}


\section{Conclusions \label{sec:conclusions}}


%--References
\small
\renewcommand{\bibsep}{0em}

\renewcommand{\bibname}{References}
\bibliographystyle{Latex/gji}
\bibliography{refs}

\end{document}
