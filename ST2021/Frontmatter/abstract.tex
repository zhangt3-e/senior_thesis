
% Senior thesis or Junior Project Abstract -----------------------------------------------------

Princeton's climate has four seasons, with strong temperature
variations, and precipitation occurring throughout the
year. Statistically, precipitation event sequences can be
characterized as drawn from exponential distributions in the three
variables \textit{duration}, \textit{intensity} (the total
precipitation divided by the duration), and interevent
\textit{separation}. The shortest and least intense precipitation
events are the most frequent. Analyzing the precipitation measured
from 2017 to the present day by a Vaisala WXT530 weather station
located on the roof of Guyot Hall, I first summarize the data in terms
of exponential distributions and their parameters, by season and by
year. Subsequently, I evaluate the skill in predicting the arrival,
duration and intensity of precipitation events solely based on this
local ``climatology'', before including other variables logged by the
weather station.
