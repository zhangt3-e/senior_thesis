
% Senior thesis or Junior Project Abstract -----------------------------------------------------

Princeton's climate has four seasons, with strong temperature
variations, and precipitation occurring throughout the
year. Statistically, precipitation event sequences can be
characterized as drawn from exponential distributions in the three
variables the precipitation event \textit{duration}, \textit{intensity} 
(the total precipitation divided by the duration), and the non-precipitation 
\textit{duration}. The shortest and least intense precipitation events 
are the most frequent. Analyzing the precipitation measured from 2017 
to the present day by a Vaisala WXT530 weather station located on the 
roof of Guyot Hall, I first summarize the data in terms
of exponential distributions and their parameters, by season and by
year. Subsequently, I evaluate the skill in predicting the arrival,
duration and intensity of precipitation events solely based on this
local ``climatology'', before including other variables logged by the
weather station. Predicting precipitation events using this climatology
yielded 2-3 \% precipitation accuracy. Thus, we proceeded to use linear
regression and decision trees regression to improve the precipitation
accuracy. Linear regression yielded 4.7 \%, while decision tree yielded 
11.9\%. Then neural networks were used in the form of LSTM, where we had 
hourly and minute inputs. The hourly input
resulted in 9.9\%, while the minute inputs resulted in 16.8\%. With each new model, 
we are able to see improvements in the accuracy of
predicting precipitation. However, there are further improvements that can be 
made with many pathways forward to improve the precipitation accuracy.  
