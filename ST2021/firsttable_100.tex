\begin{table}[htb]
  \begin{center}
    \begin{tabular}{|l|*{11}{r|}r|}
      \hline
      Season    &       \multicolumn{2}{|c|}{Annual}          & \multicolumn{2}{|c|}{Winter}& \multicolumn{2}{|c|}{Spring}  & \multicolumn{2}{|c|}{Summer} &\multicolumn{2}{|c|}{Fall}  \\
      \hline
      Year      & $\beta $ & $\alpha$  & $\beta $ & $\alpha$ & $\beta $ & $\alpha$ & $\beta $ & $\alpha$ & $\beta $ & $\alpha$\\
      \hline
      2017      & \textit{94}  & \textit{-0.32}  & NaN & NaN & NaN & NaN & \textit{32}  & \textit{-0.14}  & 68  & -0.09  \\
      2018      & 214           & -0.28  & 73 & -0.37 & 46 & -0.27 & 38  & -0.24  & 55 & -0.23  \\
      2019      & 224          & -0.28  & 68  & -0.34 & 60 & -0.28 & 25 & -0.19  & 33 &  -0.32 \\
      2020      & 234           & -0.35   & 119  & -0.51 & 60  & -0.32 & 36  & -0.27 & \textit{29} & \textit{-0.23}\\
      \hline
    \end{tabular}
  \end{center}
\caption[Yearly comparison of
  coefficients of precipitation duration using 0 to 100th percentile]{\label{firsttable_100} Coefficients found from the yearly
  distribution of precipitation duration (as in equation~\ref{expod})
  as well as the seasonal distribution of precipitation
  duration. Italics refer to values obtained using incomplete
  information. NaN means there was no information. These coefficients were computed using the all the data, from 0 to 100th percentile.}
\end{table}
